\documentclass{article}
\usepackage{geometry}
\geometry{a4paper, margin=1in}
\usepackage{amsmath}
\usepackage{graphicx}
\usepackage{hyperref}

\title{Investigation Framework for Evaluating Embedding Models in Dense Retriever Search Systems}
\author{Research Team}
\date{\today}

\begin{document}

\maketitle

\section{Introduction}
The objective of this investigation is to \textbf{evaluate the quality and consistency of text embedding models for dense retriever search systems} 
through the development and comparison of novel evaluation metrics, such as the A-QSR-RDMD. 
The investigation focuses on understanding \textbf{how embedding models behave under meaning-preserving input variations} 
and aims to assess their suitability for search tasks by comparing them across multiple evaluation measures.

\subsection{Goals and Motivation}
\begin{itemize}
    \item \textbf{Research Problem:} How well do embedding models maintain consistency and accurately represent semantic relationships in the context of search tasks?
    \item \textbf{Objective:} Develop and evaluate new measures (e.g., A-QSR-RDMD) to capture latent properties of embedding models such as \textbf{semantic consistency} and \textbf{variance in response to ambiguous inputs}.
    \item \textbf{Impact:} Inform the development of \textbf{better evaluation methodologies} for dense retrieval systems and assess \textbf{consistency} as a desirable property in NLP models.
\end{itemize}

\section{Experimental Setup}
\subsection{Embedding Models}
\begin{itemize}
    \item \textbf{Word2Vec} (via spaCy pipeline)
    \item \textbf{RoBERTa-based} model (via \texttt{sentence-transformers})
    \item Modular framework to incorporate additional models as needed (e.g., BERT, DPR).
    \item \textbf{CUDA acceleration} enabled where available.
\end{itemize}

\subsection{Search Dataset}
\begin{itemize}
    \item \textbf{MS MARCO V2}: A subset containing queries, qrels (relevance judgments), and corpus data.
    \item Dataset files:
    \begin{itemize}
        \item \textbf{Queries:} (qid, query)
        \item \textbf{Qrels:} (qid, docid, relevance)
        \item \textbf{Corpus:} (document text)
    \end{itemize}
\end{itemize}

\section{Evaluation Metrics}
\subsection{Core Metrics}
\begin{itemize}
    \item \textbf{NDCG} (Normalized Discounted Cumulative Gain)
    \item \textbf{Recall@k}
    \item \textbf{MRR} (Mean Reciprocal Rank)
\end{itemize}

\subsection{Proposed New Measures}
\begin{itemize}
    \item \textbf{A-QSR-RDMD}: Ambiguity by Query Semantic Region - Relevant Document Mean Deviation
    \begin{itemize}
        \item Captures the \textbf{invariance and semantic consistency} of embeddings for similar queries.
        \item Investigates the variance in \textbf{distance from centroids} for grouped queries and documents, revealing the degree of ambiguity in query embeddings.
    \end{itemize}
\end{itemize}

\section{Methodology}
\begin{enumerate}
    \item \textbf{Data Preparation}: 
    Load and preprocess the MS MARCO queries, qrels, and corpus. 
    Generate embeddings for queries and documents using selected models.
    \item \textbf{Experimental Execution}:
    \begin{itemize}
        \item Retrieve Documents: Perform search with query embeddings to identify top-k results.
        \item Compute Metrics: Calculate NDCG, Recall@k, and MRR for each model.
        \item Apply New Measures: Compute A-QSR-RDMD for each query and document region.
    \end{itemize}
    \item \textbf{Analysis}:
    \begin{itemize}
        \item Compare Models: Use both traditional and new metrics to assess each model’s performance.
        \item Consistency Evaluation: Analyze how the models behave under meaning-preserving input changes (e.g., small query variations).
        \item Visualization: Use plots and tables to present the distribution of metric values across models and datasets.
    \end{itemize}
\end{enumerate}

\section{Expected Outcomes}
\begin{itemize}
    \item \textbf{Validation of New Measures}: Demonstrate that metrics like A-QSR-RDMD capture unique properties of embedding models, such as \textbf{semantic consistency} and \textbf{ambiguity handling}.
    \item \textbf{Model Selection Insights}: Identify which embedding models exhibit desirable properties (e.g., consistency, stable centroid distances) for dense retrieval tasks.
    \item \textbf{Methodological Contributions}: Propose a modular, reproducible framework for evaluating embedding models using both traditional and new metrics.
\end{itemize}

\section{Next Steps}
\begin{enumerate}
    \item Finalize the code for A-QSR-RDMD and integrate other metrics into the experimental framework.
    \item Execute experiments with multiple models and datasets.
    \item Compare performance across models using traditional and novel metrics.
    \item Prepare a detailed report summarizing the results and insights from the investigation.
\end{enumerate}

\end{document}
